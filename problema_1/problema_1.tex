\documentclass[a4paper]{article}

\usepackage{amsmath}
\usepackage{amssymb,amsfonts}
\usepackage[catalan]{babel} % Language 
\usepackage{fontspec}
\usepackage[margin=2cm]{geometry}
\usepackage{graphicx}

\setlength{\parindent}{0pt}
\setlength{\parskip}{0.2cm}

\title{\textsc{APA Problemes} \\ Problema 3 Màxima versemblança}
\author{Lluc Bové \and Raúl Ibàñez \and Joan Marcè \and Aleix Trassera}
\date{}

\begin{document}

\maketitle

\textbf{Considerem un experiment aleatori en què mesurem una determinada variable aleatòria $X$, que segueix una distribució gaussiana, cosa que escrivim $X \sim N(\mu, \sigma^2)$. Prenem $N$ mesures independents de $X$ i obtenim una mostra aleatòria simple $\{x_1, ..., x_N\}$ on cada $x_n$ és una realització de $X$, per $n=1,...,N$. Es demana:}

\begin{enumerate}
\item \textbf{Escriviu la funció de densitat de probabilitat per un $x_n$ qualsevol i construïu la funció log-versemblança (negativa) de la mostra.}

$$p(x, \sigma^2, \mu) = \frac{1}{2\pi\sigma} exp(-\frac{(x - \mu)^2}{2\sigma^2})$$

\item \textbf{Trobeu els estimadors de màxima versemblança $\hat{\mu}$ i $\hat{\sigma}^2$ per $\mu$ i per $\sigma^2$, a partir de la mostra.}

\item \textbf{Demostreu que realment són màxims (i no extrems qualssevol).}

\item \textbf{Calculeu els biaixos dels dos estimadors. Determineu si l'estimador per $\mu$ és consistent.}

\item \textbf{Calculeu la variança de l'estimador per $\mu$, de 3 maneres (que han de coincidir). Pista: useu que si $X_n, X_m \sim N(\mu, \sigma^2)$, llavors:}

$$ \mathbb{E}[X_n · X_m] = 
\begin{cases}
\mu^2 & \text{ si } n \ne m \\
\mu^2 + \sigma^2 & \text{ si } n = m
\end{cases}$$

\begin{enumerate}
    \item Inserint directament el valor de l'estimador i utilitzant les propietats de la variància. 
    \item Usant la coneguda fòrmula $Var[\hat{\theta}] = \mathbb{E}[\hat{\theta}^2] - (\mathbb{E}[\hat{\theta}])^2$.
    \item Utilitzant la definició de la variància $Var[\hat{\theta}] = \mathbb{E}[(\mathbb{E}[\hat{\theta}^2] - \hat{\theta})^2]$.
\end{enumerate}

\end{enumerate}

\end{document}