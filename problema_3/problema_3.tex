\documentclass[a4paper]{article}
\usepackage[margin=2cm]{geometry}

\usepackage{fontspec}			% utf-8 suport
\usepackage{amsmath}			% Math utilities
\usepackage{amssymb}			% Math symbols
\usepackage{amsfonts}			% More math fonts
\usepackage{mathtools}			% More math tools (rcases)
\usepackage[makeroom]{cancel}	% Math cancel
\usepackage[catalan]{babel} 	% Language 
\usepackage{enumitem}			% To resume the enumeration
\usepackage{gensymb}			% º Symbol
\usepackage{newunicodechar}		% To add new chars
\newunicodechar{º}{\degree}

\setlength{\parindent}{0pt}
\setlength{\parskip}{0.2cm}

\title{\textsc{APA Problemes} \\ Problema 6 Càlcul d'òrbites}
\author{Lluc Bové \and Raúl Ibáñez \and Joan Marcè \and Aleix Trassera}
\date{}

\begin{document}
\maketitle

\textbf{El cometa Tentax es va descobrir al 1968 i té una òrbita quadràtica (e\l. líptica, parabòlica o hiperbòlica) d'acord a les lleis de Kepler. L'òrbita té l'equació:}
$$
r = \frac{p}{1 - e \cos \phi}
$$

\textbf{on $p$ és un coeficient específic per aquest cometa, $e$ és l'excentricitat (totes dues desconegudes) i les parelles ($r,\phi$) indiquen les diferents posicions observades (en coordenades polars amb centre en el Sol). Els astrònoms han reunit un conjunt de coordenades:}
$$
\{(2.70, 48º), (2:00; 67º), (1.61, 83º), (1.20, 108º), (1.02, 126º)\}
$$

\begin{enumerate}
	\item \textbf{Escriviu el problema com un sistema lineal}
\end{enumerate}

\begin{enumerate}[resume]
	\item \textbf{Trobeu les dues constants $p,e$ per mínims quadrats}
\end{enumerate}

\end{document}