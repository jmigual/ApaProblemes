\documentclass[a4paper]{article}
\usepackage[margin=2cm]{geometry}

\usepackage{fontspec}			% utf-8 suport
\usepackage{amsmath}			% Math utilities
\usepackage{amssymb}			% Math symbols
\usepackage{amsfonts}			% More math fonts
\usepackage{mathtools}			% More math tools (rcases)
\usepackage[makeroom]{cancel}	% Math cancel
\usepackage[catalan]{babel} 	% Language 
\usepackage{enumitem}			% To resume the enumeration
\usepackage{gensymb}			% º Symbol
\usepackage{newunicodechar}		% To add new chars
\newunicodechar{º}{\degree}

\setlength{\parindent}{0pt}
\setlength{\parskip}{0.2cm}

\title{\textsc{APA Problemes} \\ Problema 6 Càlcul d'òrbites}
\author{Lluc Bové \and Raúl Ibáñez \and Joan Marcè \and Aleix Trassera}
\date{}

\begin{document}
\maketitle

\textbf{El cometa Tentax es va descobrir al 1968 i té una òrbita quadràtica (e\l. líptica, parabòlica o hiperbòlica) d'acord a les lleis de Kepler. L'òrbita té l'equació:}
$$
r = \frac{p}{1 - e \cos \phi}
$$

\textbf{on $p$ és un coeficient específic per aquest cometa, $e$ és l'excentricitat (totes dues desconegudes) i les parelles ($r,\phi$) indiquen les diferents posicions observades (en coordenades polars amb centre en el Sol). Els astrònoms han reunit un conjunt de coordenades:}
$$
\{(2.70, 48º), (2:00; 67º), (1.61, 83º), (1.20, 108º), (1.02, 126º)\}
$$

\begin{enumerate}
	\item \textbf{Escriviu el problema com un sistema lineal}
\end{enumerate}

Es busca obtenir una funció de l'estil:
$$
r = \frac{p}{1 - e \cos\phi} \quad \longrightarrow \quad y(\vec{w}, \vec{x}) = \sum_{i=0}^{M - 1} w_i \phi_i(\vec{x})
$$

Així doncs es fa la següent transformació
$$
r = \frac{p}{1 - e \cos \phi} \implies \frac{1}{r} = \frac{1 - e \cos \phi}{p} 
\implies \underbrace{r^{-1}}_{t} = 
\underbrace{p^{-1}}_{w_0}·
\underbrace{1}_{\mathclap{\phi_0}} +
\underbrace{\frac{e}{p}}_{w_1}·\underbrace{(-\cos\phi)}_{\phi_1}
$$
Per tant s'obtenen els següents vectors:
$$
\vec{w} = 
\begin{pmatrix}
p^{-1}\\
e/p
\end{pmatrix}
\qquad
\vec{\phi}(\vec{x}) = \vec{\phi}(\phi) =
\begin{pmatrix}
1\\
- \cos \phi 
\end{pmatrix}
$$

Així doncs s'obté la següent matriu de disseny:
$$
\Phi = 
\begin{pmatrix}
1 & -0.64 \\
1 & -0.52 \\
1 & 0.25 \\
1 & 0.38 \\
1 & 0.94 \\
\end{pmatrix}
\qquad
t =
\begin{pmatrix}
0.37 \\
0.50 \\
0.62 \\
0.83 \\
0.98 \\
\end{pmatrix}
$$
I el sistema lineal que s'obté és:
$$
(\Phi^T\Phi)w = \Phi^T t \implies w = (\Phi^T\Phi)^{-1} \Phi^T t 
$$
\begin{enumerate}[resume]
	\item \textbf{Trobeu les dues constants $p,e$ per mínims quadrats}
\end{enumerate}

El vector $\vec{w}$ conté: $w[1] = p^{-1}$ i $ w[2] = \frac{e}{p}$. Per tant:

$ p = q / w[1] = 1.583654 $

$ e = p * w[2] = 0.5699404 $


\end{document}