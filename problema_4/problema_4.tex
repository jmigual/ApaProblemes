\documentclass[a4paper]{article}
\usepackage[margin=2cm]{geometry}

\usepackage{fontspec}			% utf-8 suport
\usepackage{amsmath}			% Math utilities
\usepackage{amssymb}			% Math symbols
\usepackage{amsfonts}			% More math fonts
\usepackage{mathtools}			% More math tools (rcases)
\usepackage[makeroom]{cancel}	% Math cancel
\usepackage[catalan]{babel} 	% Language 
\usepackage{enumitem}			% To resume the enumeration
\usepackage{gensymb}			% º Symbol
\usepackage{newunicodechar}		% To add new chars
\usepackage{pgfplots}

\newunicodechar{º}{\degree}

\setlength{\parindent}{0pt}
\setlength{\parskip}{0.2cm}

\title{\textsc{APA Problemes} \\ Problema 7 La fàbrica de píndoles III}
\author{Lluc Bové \and Raúl Ibáñez \and Joan Marcè \and Aleix Trassera}
\date{}

\begin{document}

\maketitle

\textbf{La companyia de farmacèutica \emph{Smart Pills} (competidora de les anteriors) ha construït una cinta transportadora que porta dues \emph{classes} de píndoles (adequades per dos tipus de malalties diferents), que anomenem $C_1$ i $C_2$. Aquestes píndoles surten en un ombrejat de colors que va del \emph{yellow} al \emph{white} (que és detectat per una càmera, donant un valor continu en $[0,2]$). La companyia fabrica píndoles en proporcions $P(C_1) = \frac{1}{3},\ P(C_2) = \frac{2}{3}$. Se'ns facilita també informació sobre la distribució (contínua) del color per cada classe: }
$$
p(x|C_1) = \frac{2 - x}{2} \qquad p(x|C_2) = \frac{x}{2}
$$

\begin{enumerate}
	\item \textbf{Quina és la probabilitat d'error si no s'utilitza el color per classificar?}
\end{enumerate}

Si no s'usa el color per classificar directament es dirà que totes les píndoles són de classe $C_2$ (ja que la probabilitat és més alta). Així doncs la probabilitat d'error és que sigui una píndola de classe $C_1$.

$$
P(error) = P(C_1) = \frac{1}{3}
$$

\begin{enumerate}[resume]
	\item \textbf{Calcular la distribució \emph{incondicional} de color $p(x) = P(C_1)P(x|C_1) + P(C_2)P(x|C_2)$.}
\end{enumerate}
$$
p(x) = P(C_1)P(x|C_1) + P(C_2)P(x|C_2) = \frac{1}{3}·\frac{2 - x}{2} + \frac{2}{3}·\frac{x}{2} = \frac{2 - x}{6} + \frac{2x}{6} = \boxed{\frac{x + 2}{6}}
$$

\begin{enumerate}[resume]
	\item \textbf{Calcular les distribucions de probabilitat $P(C_1|x)$ i $P(C_2|x)$.}
\end{enumerate}
$$
P(C_1|x) = \frac{P(x|C_1)P(C_1)}{P(x)} = 
\frac{\frac{2 - x}{2}·\frac{1}{3}}{\frac{x + 2}{6}} =
\frac{\frac{2 - x}{6}}{\frac{x + 2}{6}} = \boxed{\frac{2 - x}{x + 2}}
$$
$$
P(C_2|x) = 1 - P(C_1|x) = \frac{x + 2}{x + 2} - \frac{2 - x}{x + 2} =
\boxed{\frac{2x}{x + 2}}
$$

\begin{enumerate}[resume]
	\item \textbf{Quina és la classificació òptima en funció del color?}
\end{enumerate}

\begin{tikzpicture}
	
\end{tikzpicture}

\begin{enumerate}[resume]
	\item \textbf{Quina és la probabilitat d'error si s'utilitza el color per classificar? Per què és millor que la de l'apartat 1?}
\end{enumerate}









\end{document}
