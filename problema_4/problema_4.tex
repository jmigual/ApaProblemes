\documentclass[a4paper]{article}
\usepackage[margin=2cm]{geometry}

\usepackage{fontspec}			% utf-8 suport
\usepackage{amsmath}			% Math utilities
\usepackage{amssymb}			% Math symbols
\usepackage{amsfonts}			% More math fonts
\usepackage{mathtools}			% More math tools (rcases)
\usepackage[makeroom]{cancel}	% Math cancel
\usepackage[catalan]{babel} 	% Language 
\usepackage{enumitem}			% To resume the enumeration
\usepackage{gensymb}			% º Symbol
\usepackage{newunicodechar}		% To add new chars
\usepackage{tikz}				% Better drawing options
\usepackage{pgfplots}			% To add graphs
\usepackage{float}				% Better positioning with figures
\usepackage[hidelinks]{hyperref}
\pgfplotsset{compat=1.13}

\usetikzlibrary{patterns}

\newunicodechar{º}{\degree}

\setlength{\parindent}{0pt}
\setlength{\parskip}{0.2cm}

\title{\textsc{APA Problemes} \\ Problema 7 La fàbrica de píndoles III}
\author{Lluc Bové \and Raúl Ibáñez \and Joan Marcè \and Aleix Trassera}
\date{}

\begin{document}

\maketitle

\textbf{La companyia de farmacèutica \emph{Smart Pills} (competidora de les anteriors) ha construït una cinta transportadora que porta dues \emph{classes} de píndoles (adequades per dos tipus de malalties diferents), que anomenem $C_1$ i $C_2$. Aquestes píndoles surten en un ombrejat de colors que va del \emph{yellow} al \emph{white} (que és detectat per una càmera, donant un valor continu en $[0,2]$). La companyia fabrica píndoles en proporcions $P(C_1) = \frac{1}{3},\ P(C_2) = \frac{2}{3}$. Se'ns facilita també informació sobre la distribució (contínua) del color per cada classe: }
$$
p(x|C_1) = \frac{2 - x}{2} \qquad p(x|C_2) = \frac{x}{2}
$$

\begin{enumerate}
	\item \textbf{Quina és la probabilitat d'error si no s'utilitza el color per classificar?}
\end{enumerate}

Si no s'usa el color per classificar directament es dirà que totes les píndoles són de classe $C_2$ (ja que la probabilitat és més alta). Així doncs la probabilitat d'error és que sigui una píndola de classe $C_1$.

$$
P_{error}(R_1) = \min(P(C_1),P(C_2)) = P(C_1) = \frac{1}{3}
$$

Així doncs la regla de classificació serà la següent:
$$
R_1: \text{ la classe de } x =
\begin{cases}
C_1 \text{ si } P(C_1) > P(C_2) \\
C_2 \text{ si } P(C_1) < P(C_2)
\end{cases}
$$

\begin{enumerate}[resume]
	\item \textbf{Calcular la distribució \emph{incondicional} de color $p(x) = P(C_1)P(x|C_1) + P(C_2)P(x|C_2)$.}
\end{enumerate}
$$
p(x) = P(C_1)P(x|C_1) + P(C_2)P(x|C_2) = \frac{1}{3}·\frac{2 - x}{2} + \frac{2}{3}·\frac{x}{2} = \frac{2 - x}{6} + \frac{2x}{6} = \boxed{\frac{x + 2}{6}} 
$$

\begin{enumerate}[resume]
	\item \textbf{Calcular les distribucions de probabilitat $P(C_1|x)$ i $P(C_2|x)$.}
\end{enumerate}
$$
P(C_1|x) = \frac{P(x|C_1)P(C_1)}{P(x)} = 
\frac{\frac{2 - x}{2}·\frac{1}{3}}{\frac{x + 2}{6}} =
\frac{\frac{2 - x}{6}}{\frac{x + 2}{6}} = \boxed{\frac{2 - x}{x + 2}}
$$
$$
P(C_2|x) = 1 - P(C_1|x) = \frac{x + 2}{x + 2} - \frac{2 - x}{x + 2} =
\boxed{\frac{2x}{x + 2}}
$$

\begin{enumerate}[resume]
	\item \textbf{Quina és la classificació òptima en funció del color?}
\end{enumerate}
En la \autoref{fig:regla} es pot observar que la regla a definir per a la classificació òptima és la següent:

$$
R_{bayes}: \text{ la classe de } x = 
\begin{cases}
C_1 \text{ si } P(C_1|x) > P(C_2|x) \\
C_2 \text{ si } P(C_1|x) < P(C_2|x)
\end{cases}
=
\begin{cases}
C_1 \text{ si } \frac{2 - x}{x + 2} > \frac{2x}{x + 2} \\
C_2 \text{ si } \frac{2 - x}{x + 2} < \frac{2x}{x + 2} \\
\end{cases}
$$

\begin{figure}[H]
	\centering
	\begin{tikzpicture}
		\pgfdeclarepatternformonly{flexible hatch}%
		{\pgfqpoint{-1pt}{-1pt}}%
		{\pgfqpoint{10pt}{10pt}}%
		{\pgfqpoint{6pt}{6pt}}%
		{
			\pgfsetlinewidth{1pt}
			\pgfpathmoveto{\pgfqpoint{6.1pt}{0pt}}
			\pgfpathlineto{\pgfqpoint{0pt}{6.1pt}}
			\pgfusepath{stroke}
		}
		\pgfdeclarepatternformonly{flexible inverted}%
		{\pgfqpoint{-1pt}{-1pt}}%
		{\pgfqpoint{10pt}{10pt}}%
		{\pgfqpoint{6pt}{6pt}}%
		{
			\pgfsetlinewidth{1pt}
			\pgfpathmoveto{\pgfqpoint{0pt}{0pt}}
			\pgfpathlineto{\pgfqpoint{6.1pt}{6.1pt}}
			\pgfusepath{stroke}
		}
		\begin{axis}[
				width=0.7\textwidth,height=0.7\textwidth,
				xmin=0,xmax=2.1,ymin=0,ymax=1.1,
				domain=0:2,
				axis x line=center,
				axis y line=center,
				axis on top,
				samples = 200,
				legend cell align = left,
				legend style={at={(0.9,0.5)},anchor=east},
			]
			\addplot+[
				black,
				mark=none,
				dashed,
				very thick]{max((2*x)/(x + 2),(2 - x)/(x + 2))};
			\addplot[
				mark=none, 
				forget plot, 
				black, 
				dashed] coordinates {(2/3,0)(2/3,2)};
			\addplot[
				white,
				mark = none,
				area legend,
				pattern = flexible hatch,
				pattern color = black!20!green!40] {1/3} \closedcycle;
			\addplot[
				white,
				mark=none,
				area legend, 
				pattern = flexible inverted,	
				pattern color = orange!40] {min((2*x)/(x + 2),(2 - x)/(x + 2))} \closedcycle;
				\addplot[mark=none, orange]{1/3};
				\addplot[mark=none, purple]{2/3};
				\addplot[mark=none, red] {(2 - x)/(x + 2)};
				\addplot[mark=none, blue] {(2*x)/(x + 2)};
			\legend{$R_{bayes}$, $ \mathbb{E}(P_{error}(R_1))$,  $\mathbb{E}(P_{error}(R_{bayes}))$,  $P(C_1|X)$, $P(C_2|X)$, $P(C_1)$,$P(C_2)$};
		\end{axis}
	\end{tikzpicture}
	\caption{Gràfic amb la regla per classificar en funció del color i la probabilitat d'error}
	\label{fig:regla}
\end{figure}

\begin{enumerate}[resume]
	\item \textbf{Quina és la probabilitat d'error si s'utilitza el color per classificar? Per què és millor que la de l'apartat 1?}
\end{enumerate}

Si s'utilitza el color per classificar, la probabilitat d'error és:
$$
P_{error}(R_{bayes}) = \min(P(C_1|X),P(C_2|X)) = 
\boxed{\min\left(\frac{2 - x}{x + 2},\frac{2x}{x + 2}\right)}
$$

L'error total s'ha d'integrar l'àrea sota la corba de les probabilitats $P(C_1|x)$ i $P(C_2|x)$, àrea ombrejada de la \autoref{fig:regla}. És a dir s'ha d'integrar $\min(P(C_1|X),P(C_2|x))$. Es farà una integral per trossos però primer s'ha de buscar el punt d'intersecció de les corbes, és a dir, el punt amb la mateixa probabilitat tant per $C_1$ com per $C_2$.
$$
\mathbb{E}(P_{error}(R_{bayes})) = \int \min(P(C_1|x),P(C_2|x)) dx
$$
$$
P(C_1|x) = P(C_2|x) \implies \frac{2 - x}{x + 2} = \frac{2x}{x + 2} \implies
2 - x = 2x \implies \boxed{x = \frac{2}{3}}
$$
\begin{align*}
& \int \min(P(C_1|x),P(C_2|x)) dx = 
\int \min\left(\frac{2 - x}{x + 2},\frac{2x}{x + 2}\right) dx =
\int\limits_{0}^{\frac{2}{3}} \frac{2x}{x + 2} dx + 
\int\limits_{\frac{2}{3}}^{2} \frac{2 - x}{x + 2} dx = \\
& = \cancel{\frac{4}{3}} + 2\ln\left(\frac{9}{16}\right) + 
4\ln\left(\frac{3}{2} \right) - \cancel{\frac{4}{3}} = 
2\ln\left(\frac{9}{16}\right) + 2\ln\left( \frac{9}{4} \right) =
\boxed{2\ln\left(\frac{81}{64}\right) = \underline{\underline{0,471}}}
\end{align*}
\begin{align*}
\textbullet 
&\int\limits_{0}^{\frac{2}{3}} \frac{2x}{x + 2} dx =
\begin{rcases}
\begin{cases}
u - 2 &= x  \\
du &= dx
\end{cases}
\end{rcases} =
2 \int\limits_{2}^{\frac{8}{3}} \frac{u - 2}{u - \cancel{2} + \cancel{2}} du = 
2 \int\limits_{2}^{\frac{8}{3}}  1 -\frac{2}{u} du = \\
&= 2 [u - 2\ln(u)]_2^{\frac{8}{3}} = 
2 \left( \frac{8}{3} - 2\ln\left( \frac{8}{3} \right) \right) - 2 (2 - 2\ln(2)) =
% Resultat primera integral
\underline{\underline{\frac{4}{3} + 2\ln\left(\frac{9}{16}\right)}} \\
% Segona integral
\textbullet & 
\int\limits_{\frac{2}{3}}^{2} \frac{2 - x}{x + 2} = 
\begin{rcases}
\begin{cases}
u - 2 &= x  \\
du &= dx
\end{cases}
\end{rcases} =
\int\limits_{\frac{8}{3}}^4 \frac{2 - (u - 2)}{u - \cancel{2} + \cancel{2}} du =
\int\limits_{\frac{8}{3}}^4 \frac{4 - u}{u} du =
\int\limits_{\frac{8}{3}}^4 \frac{4}{u} - 1 \ du = \\
&= [4\ln(u) - u]_{\frac{8}{3}}^4 = 
(4\ln(4) - 4) - \left( 4\ln\left( \frac{8}{3}\right) - \frac{8}{3} \right) =
% Resultat segona integral
\underline{\underline{4\ln\left(\frac{3}{2} \right)- \frac{4}{3}}}
\end{align*}

L'error total amb la primera regla és:
$$
\mathbb{E}(P_{error}(R_1)) = \int\limits_{0}^{2} P(C_1)dx = 
\int\limits_{0}^{2}\frac{1}{3}dx = \boxed{\frac{2}{3} = 0.667}
$$

Com es pot veure l'error total és inferior quan s'utilitza el color que quan no s'utilitza el color.
$$
0.667 < 0.471 \implies 
\mathbb{E}(P_{error}(R_1)) > \mathbb{E}(P_{error}(R_{bayes}))
$$





\end{document}
